\section{Le test}


\subsection{Introduction}
On va faire des test unitaire pour tester le programme. Ils testent une partie du programme de manière indépendante des autres.
\\On prend une classe , on l’isole et on vérifie qu’elle fait bien ce qu’elle est censé faire.
\\Il existe différentes approches de test:
\begin{description}
	\item [Test exploratoire] On joue avec l’application, on utilise l’interface, on a rien prévu pour le test. Utiliser des outils qui injectent des données à l’application.
	\item [Test unitaire] On ne teste pas l’application mais les composants. On va créer un code supplémentaire pour tester.
	\item [Stub ou Mock] Remplacer une classe par une autre, minimaliste et ne fournissant que l’interface ainsi éventuellement des outillages de test.
\end{description}


\subsection{Conditions}
\begin{description}
	\item [Invariant] Expression booléenne qui est toujours vraie. On commence donc par les vérifier. Si un des invariant est faux, c’est qu’il y a une erreur.
	\item [Pré-condition et Post-conditions] Expression assurées être vraies avant ou après la méthode.
\end{description}


\subsection{Assertions}:
\begin{description}
	\item [Assertion] Permet d’exprimer une expression booléenne qui doit être vraie à cet endroit.
	\\Cela permet de formaliser et d’écrire des pré-conditions, invariants et post-conditions.
	\\Avec une assertion, si l’invariant est violé on sait où et l’erreur ne se propage pas.
\end{description}


\subsection{Test unitaire}
	On écrit en général les test unitaires sous la forme d’objet. On les appelles des testssuites. Un avantage des tests unitaires est qu’on peut les relancer très facilement. On peut rejouer les test même sur ce qu’on a pas modifié. On vérifie qu’on a pas cassé quelque chose qui marchait déjà. C’est un test de non-régression.


\subsubsection{Rédaction de tests}
	Les développeurs doivent écrire du code et des tests en même temps mais parfois on écrit d’abord les tests puis le code.
	\\\textbf{Extreme programming}: Technique de programmation agile, très fortement itérative qui se base la dessus.
	\\Écrire d’abord les test permet de tester la spécification et non l’implémentation.
	\\Avant on écrivait beaucoup de documentation, qui finissait par ne plus être à jour. La meilleure source d’information sur une application est le code. Il faut investir du temps au niveau du code, de son auto-documentation, ses tests, sa lisibilité, ...
	\\Les tests servent maintenant de documentation car ils sont toujours à jour.


\subsubsection{Méthodologie de rédaction de tests}
	Si on a un bug trouvé par le client, c’est qu’il est passé à travers les tests, donc aucun ne le couvre.
	\\On rédige alors un test qui échoue sur le bug et seulement ensuite on va pouvoir le corriger.
	\\Si en cherchant le bug on place des printf, c’est que ça devrait se retrouver dans le test.


\subsection{Mise en place de tests unitaires}
	On va tester une classe ou un petit morceau de programme.
	\\Il existe des frameworks de test unitaires dans presque tout les langages. ( CppUnit pour C++)
	\\En CppUnit, un testcase hérite de CppUnit::TestCase. La classe du test devrai avoir le nom “NomTest”.
	\\Un échec est une assertion qui rate, mais on l’avait anticipé. Une Erreur est un bug non-détecté par une assertion.


\subsubsection{Qualité des tests unitaires}
	Un bon test unitaire doit avoir certaines propriétés.
	\\Un test unitaire doit être déterministe: A chaque lancement du test, on doit avoir le même résultat. Sinon on peut perdre confiance dans les tests. Le lancement doit être automatisé.
	\\Les test sont une source de documentation. 
	Ils doivent donc être lisible. Ils doivent être le moins sensible possible aux changements du code.
	\\Ils ne doivent pas tester des choses évidentes mais traiter les choses complexes et difficiles en premier.


\subsubsection{Frameworks de test}
	Les éléments clé d’un framework de tests sont le TestCase ( Cas de test ), les mécanisme pour les exécuter, les fixtures (un contexte, les objets cobaye dont on a besoin, les TestSuites ( un ensemble de TestCase) et le TestRunner qui va déclencher les tests.

	On commence par déclarer un TestRunner. Un Outputter prend le flux de sortie et le met ailleurs. On enregistre les test auprès du TestRunner. Puis on le lance et l’Outputter contient maintenant les résultats.

	\textbf{TestFixture}: Une fixture est le contexte partagé par tout les testcase d’une suite de test.
	\\Entre chaque test, le contexte est réinitialisé à la fixture de départ.